
Я был очень мудрым стариком.

Теперь я уже не то, считайте даже, что меня нет. Но было время, когда любой из вас
пришел бы ко мне, и какая бы тяжесть ни томила его душу, какие бы грехи ни терзали его
мысли, я бы обнял его и сказал: <<Сын мой, утешься, ибо никакая тяжесть души твоей не
томит и никаких грехов не вижу я в теле твоем>>, и он убежал бы от меня счастливый и
радостный.

Я был велик и силен. Люди, встречая меня на улице, шарахались в сторону, и я проходил
сквозь толпу, как утюг.

Мне часто целовали ноги, но я не протестовал, я знал, что достоин этого. Зачем лишать людей радости почтить меня? Я даже сам,
будучи чрезвычайно гибким в теле, попробовал поцеловать себе свою собственную ногу. Я сел
на скамейку, взял в руки свою правую ногу и подтянул ее к лицу. Мне удалось поцеловать
большой палец на ноге. Я был счастлив. Я понял счастье других людей.

Все преклонялись передо мной! И не люди, даже звери, даже разные букашки ползали передо мной и виляли своими хвостами. 
А кошки! Те просто души во мне не чаяли и, каким"=то образом сцепившись лапами друг с другом,
бежали передо мной, когда я шел по лестнице.

В то время я был действительно очень мудр и все понимал. Не было такой вещи, перед которой я встал бы в тупик. Одна минута
напряжения моего чудовищного ума "--- и самый сложный вопрос разрешался наипростейшим образом. Меня даже водили в Институт мозга и
показывали ученым профессорам. Те электри чеством измерили мой ум и просто опупели. <<Мы
никогда ничего подобного не видали>>, "--- сказали они.

Я был женат, но редко видел свою жену. Она боялась меня: колосальность моего ума
подавляла ее. Она не жила, а трепетала, и если я смотрел на нее, она начинала икать.
Мы долго жили с ней вместе, но потом она, кажется, куда"=то исчезла: точно не помню.

Память "--- это вообще явление странное. Как трудно бывает что"=нибудь запомнить и
как легко забыть! А то и так бывает: запомнишь одно, а вспомнишь совсем другое. Или:
запомнишь что"=нибудь с трудом, но очень крепко, и потом ничего вспомнить не сможешь.
Так тоже бывает. Я бы всем советовал поработать над своей памятью.

Я был всегда справедлив и зря никого не бил, потому что, когда кого"=нибудь бьешь,
то всегда жалеешь, и тут можно переборщить. Детей, например, никогда не надо бить ножом
или вообще чем"=нибудь железным. А женщин, наоборот: никогда не следует бить ногой.
Животные, те, говорят, выносливее. Но я производил в этом направлении опыты и знаю,
что это не всегда так.

Благодаря своей гибкости я мог делать то, чего никто не мог сделать. Так, например, мне удалось однажды достать рукой из
очень извилистой фановой трубы заскочившую туда серьгу моего брата. Я мог, например, спрятаться в сравнительно небольшую корзинку и закрыть за собой крышку.

Да, конечно, я был феноменален!

Мой брат был полная моя противополож ность: во"=первых, он был выше ростом, а во"=вторых, "--- глупее.
 
Мы с ним никогда не дружили. Хотя, впрочем, дружили, и даже очень. Я тут что"=то напутал: мы именно с ним не дружили и всегда
были в ссоре. А поссорились мы с ним так. Я стоял: там выдавали сахар, и я стоял в очереди и старался не слушать, что говорят кругом. 
У меня немножечко болел зуб, и настроение было неважное. На улице было очень холодно, потому что все стояли в ватных шубах и все"=таки мерзли. 
Я тоже стоял в ватной шубе, но сам не очень мерз, а мерзли мои руки, потому что то и дело приходилось вынимать их
из кармана и поправлять чемодан, который я держал, зажав ногами, чтобы он не пропал.
Вдруг меня ударил кто"=то по спине. Я пришел в неописуемое негодование и с быстротой молнии стал обдумывать, как наказать обидчика.
В это время меня ударили по спине вторично. Я весь насторожился, но решил голову назад не поворачивать и сделать вид, будто я ничего не заметил. 
Я только на всякий случай взял чемодан в руку. Прошло минут семь, и меня в третий раз ударили по спине. Тут я повернулся и увидел перед собой 
высокого пожилого человека в довольно поношенной, но все же хорошей ватной шубе.

"--*Что вам от меня нужно? "--- спросил я
его строгим и даже слегка металлическим голосом.

"--* А ты чего не оборачиваешься, когда тебя окликают? "--- сказал он.
 
Я задумался над смыслом его слов, когда он опять открыл рот и сказал:

"--* Да ты что? Не узнаешь, что ли, меня? Ведь я твой брат.
Я опять задумался над его словами, а он снова открыл рот и сказал:

"--* Послушай"=ка, брат. У меня не хватает
на сахар четырех рублей, а из очереди уходить обидно. Одолжи"=ка мне пятерку, и мы с
тобой потом рассчитаемся.
 
Я стал раздумывать о том, почему брату не хватает четырех рублей, но он схватил меня за рукав и сказал:

"--* Ну так как же: одолжишь ты своему брату немного денег? "--- И с этими словами он сам растегнул мне мою ватную шубу, залез ко мне во внутренний карман и достал мой кошелек.
 
"--* Вот, "--- сказал он, "--- я, брат, возьму у тебя взаймы некоторую сумму, а кошелек, вот смотри, я кладу тебе обратно в пальто. "--- И он сунул кошелек в наружный карман моей шубы.
 
Я был, конечно, удивлен, так неожиданно
встретив своего брата. Некоторое время я помолчал, а потом спросил его:
 
"--* А где же ты был до сих пор?
 
"--* Там, "--- отвечал мне брат и показал куда"=то рукой.
 
Я задумался: где это <<там>>, но брат подтолкнул меня в бок и сказал:

"--* Смотри: в магазин начали пускать.
 
До дверей магазина мы шли вместе, но в магазине я оказался один, без брата. Я на минуту выскочил из очереди и выглянул через
дверь на улицу. Но брата нигде не было.
 
Когда я хотел опять занять в очереди свое место, меня туда не пустили и даже постепенно вытолкали на улицу. Я сдерживая
гнев на плохие порядки, отправился домой. Дома я обнаружил, что мой брат изъял из моего кошелька все деньги. Тут я страшно рассердился на брата, и с тех пор мы с ним никогда больше не мирились.
 
Я жил один и пускал к себе только тех, кто приходил ко мне за советом. Но таких было много, и выходило так, что я ни днем, ни
ночью не знал покоя. Иногда я уставал до такой степени, что ложился на пол и отдыхал.
Я лежал на полу до тех пор, пока мне не делалось холодно, тогда я вскакивал и начинал
бегать по комнате, чтобы согреться. Потом я опять садился на скамейку и давал советы
всем нуждающимся.

Они входили ко мне друг за другом, иногда даже не открывая дверей. Мне было весело
смотреть на их мучительные лица. Я говорил с ними, а сам едва сдерживал смех.

Один раз я не выдержал и рассмеялся. Они с ужасом кинулись бежать, кто в дверь, кто в
окно, а кто и прямо сквозь стену.

Оставшись один, я встал во весь свой могучий рост, открыл рот и сказал:

"--* Принтимпрам.

Но тут во мне что"=то хрустнуло, и с тех
пор, можете считать, что меня больше нет.